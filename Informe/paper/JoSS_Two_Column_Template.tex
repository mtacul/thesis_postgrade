
\documentclass[11pt]{article}
\usepackage[utf8]{inputenc}
\usepackage[margin=1in]{geometry}
\usepackage{amsmath, amssymb}
\usepackage{graphicx}
\usepackage{multicol}
\usepackage{titlesec}
\usepackage{fancyhdr}
\usepackage{hyperref}

% Custom section formatting
\titleformat{\section}{\normalfont\Large\bfseries}{\thesection}{1em}{}
\titleformat{\subsection}{\normalfont\large\bfseries}{\thesubsection}{1em}{}
\titleformat{\subsubsection}{\normalfont\normalsize\bfseries}{\thesubsubsection}{1em}{}

% Footer style
\pagestyle{fancy}
\fancyhf{}
\fancyfoot[L]{Copyright © A. Deepak Publishing. All rights reserved.}
\fancyfoot[C]{\thepage}
\fancyfoot[R]{JoSS, Vol. 11, No. 2, p. 1143}

% Title and author information
\title{\textbf{Ground-Based 1U CubeSat Robotic Assembly Demonstration}}
\author{\textbf{Your Name}, \textbf{Co-Author Name} \\
Department of Aeronautics and Astronautics \\
Your Institution \\
\texttt{your.email@example.com}}
\date{}

% Begin document
\begin{document}
% Title Page with Logos
\begin{titlepage}
	\begin{center}
		\vspace*{-1cm}
		\includegraphics[width=0.15\textwidth]{example-logo-left.png}
		\hfill
		\includegraphics[width=0.2\textwidth]{example-logo-right.png} \\[0.5cm]
		
		\small
		Uzo-Okoro. E. et al. (2022): JoSS, Vol. 11, No. 2, pp. 1143--1163 \\[0.2cm]
		\textit{(Peer-reviewed article available at www.jossonline.com)} \\[1cm]
		
		{\Large \textbf{Ground-Based 1U CubeSat Robotic Assembly Demonstration}} \\[0.8cm]
		
		\textbf{Your Name, Co-Author Name, and Another Author} \\[0.3cm]
		Department of Aeronautics and Astronautics \\ 
		Your Institution \\
		\textit{City, State, Country} \\[1.5cm]
		
		\textit{Published in the Journal of Small Satellites (JoSS)} \\[0.2cm]
		\textit{Copyright © A. Deepak Publishing. All rights reserved.}
	\end{center}
\end{titlepage}


% Abstract
\begin{abstract}
In-space assembly of small satellites could enable rapid response and reconfiguration of swarms and constellations. This is a sample abstract formatted based on JoSS-like styles.
\end{abstract}

% Two-column text starts from Introduction
\begin{multicols}{2}
\section{Introduction}
Provide a detailed introduction about your work. Include citations~\cite{example}. Discuss the state-of-the-art methods and the objectives of your research.

\section{Methodology}
Describe your experimental or theoretical methods in detail. Use equations as needed:
\begin{equation}
F = ma
\end{equation}

\subsection{Experimental Setup}
Explain the setup used in your study. Add figures if necessary.

\begin{figure}[h!]
\centering
\includegraphics[width=0.8\linewidth]{example-image}
\caption{Caption for your figure.}
\label{fig:example}
\end{figure}

\section{Results and Discussion}
Present and interpret your findings. Highlight key observations and compare them with prior studies.

\section{Conclusion}
Summarize the major findings and implications of your work. Suggest potential directions for future research.

\section*{Acknowledgements}
Acknowledge any support or contributions.

\end{multicols}

% References
\bibliographystyle{ieeetr}
\bibliography{references}

\end{document}
