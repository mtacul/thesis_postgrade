\section{Estado del Arte}

En este capítulo se muestran herramientas que logran simular la dinámica orbital y la dinámica de actitud del satélite, además de visualizarlo mediante una interfaz gráfica. Algunos de ellos son simuladores que entregan alguno de los MoP de apuntamiento, así como también softwares especializados en el análisis de budgets de ingeniería, en los cuales se encuentran los SE envelopes descritos.

\subsection{Spacecraft Control Toolbox}

Spacecraft Control Toolbox (SCT) para MATLAB le permite diseñar, analizar y simular naves espaciales. Este producto es utilizado en todo el mundo por organizaciones líderes en investigación y desarrollo y fabricantes de naves espaciales. Se proporcionan más de dos mil funciones para dinámica, simulación, análisis y diseño de actitud y órbita. Puedes construir un satélite utilizando las herramientas gráficas CAD; diseñar y analizar los sistemas de control; realizar análisis de perturbaciones y pruebe el sistema de control en una simulación de seis grados de libertad, todo en el lenguaje de programación MATLAB \cite{ref18}.


\subsection{Ansys Systems Tool Kit (STK)}

STK es una plataforma de software líder en el mercado diseñada para el modelado y análisis de sistemas complejos y sus interacciones en una variedad de dominios, incluyendo espacio, defensa y aplicaciones aeroespaciales. Esta herramienta proporciona una serie de capacidades avanzadas que permiten a los ingenieros, científicos y analistas modelar, simular y visualizar sistemas dinámicos de manera efectiva \cite{ref34}.

\subsection{Aerospace Blockset}

Dentro de este módulo en MATLAB, existen librerías capaces de modelar, simular y analizar CubeSats con facilidad \cite{ref19}. Algunas de las capacidades clave incluyen:

\subsection{Valispace}

Valispace es una plataforma integral de ingeniería y gestión de proyectos diseñada para simplificar y optimizar el proceso de desarrollo de productos y sistemas, especialmente en industrias como la aeroespacial. La plataforma ofrece una variedad de herramientas y características poderosas que permiten a los equipos de ingeniería colaborar eficientemente, gestionar requisitos y parámetros críticos, y tomar decisiones informadas en tiempo real \cite{ref35}.

Las características claves son la gestión de datos en tiempo real, permitiendo el acceso en tiempo real y la colaboración de los miembros del equipo de ingeniería de forma actualizada. También permite a los equipos diseñar, analizar y optimizar sistemas de ingeniería complejos en base a requisitos que se imponen en el mismo programa al inicio del proyecto. Además, Valispace facilita el cálculo de parámetros críticos como costos, masa, potencia y tamaño, lo que es esencial en proyectos de ingeniería. Los resultados se pueden calcular y actualizar automáticamente a medida que se realizan cambios en el diseño.

\subsection{Basilisk}

El simulador Basilisk es un entorno de simulación avanzado, modular y extensible, desarrollado principalmente por el Laboratorio de Sistemas de Vehículos Espaciales de la Universidad de Colorado, Boulder. Su principal objetivo es facilitar la simulación de sistemas de naves espaciales, con un enfoque en la dinámica y el control de actitud. Este simulador ha sido ampliamente utilizado en investigaciones académicas y proyectos que requieren la modelación precisa de la dinámica y control de vehículos espaciales \cite{ref36}.

Una de las principales ventajas de Basilisk es su capacidad para simular sistemas multi-plataforma y multi-cuerpo, permitiendo la modelación de la dinámica orbital y de actitud de diversos cuerpos en el espacio. Esto incluye la simulación de perturbaciones ambientales, así como el comportamiento de subsistemas complejos como el ADCS (Attitude Determination and Control System). Estas capacidades hacen que Basilisk sea especialmente útil para misiones que involucran satélites pequeños o CubeSats, donde las dinámicas precisas y los ajustes de actitud son críticos.

Otra de las fortalezas del simulador radica en su arquitectura modular. Cada subsistema —como los sensores, actuadores, o la dinámica orbital— está diseñado de manera independiente, lo que permite no solo la fácil personalización del código, sino también la posibilidad de integrar nuevos módulos o realizar simulaciones específicas de determinados componentes. Esta flexibilidad es clave para proyectos de investigación que requieren ajustes finos y configuraciones personalizadas.

El simulador también ofrece la capacidad de realizar simulaciones tanto en tiempo real como en tiempo no real, lo que resulta útil para diferentes tipos de aplicaciones. Las simulaciones en tiempo real permiten la implementación de pruebas de hardware-in-the-loop, mientras que las simulaciones en tiempo no real ofrecen mayor fidelidad para estudios más detallados.