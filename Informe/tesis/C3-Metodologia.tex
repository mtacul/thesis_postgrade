\section{Metodología}

La sección de metodología describe el diseño de investigación y los procedimientos utilizados, permitiendo a los lectores evaluar la validez y confiabilidad del estudio. Sirve como un plano detallado de cómo se llevó a cabo la investigación, incluyendo la selección de los casos de estudio, métodos de recolección de datos y técnicas de análisis. Estos métodos pueden incluir experimentos, simulaciones o levantamiento de datos de otras fuentes externas.

Se comienza describiendo el diseño general de la investigación, indicando cómo este diseño se enfoca a cumplir los objetivos y detallando sus alcances. Junto a esto, se presenta el o los casos de estudio, indicando sus características y limitaciones.

A continuación, se explican claramente los métodos utilizados para generar los datos, como experimentos, simulaciones o investigación sobre datos históricos. Se debe ser específico sobre las herramientas, instrumentos o tecnologías utilizadas y justificar por qué estos métodos fueron elegidos sobre otros.  Luego, se describen los procedimientos utilizados para analizar los datos recolectados. Especialmente para estudios experimentales, esto podría incluir análisis estadístico como pruebas de hipótesis, estimación de intervalos o análisis de varianza. 

Uno de los principales objetivos de la sección de metodología es asegurar la replicabilidad de los resultados. Esto quiere decir, que cualquier otra persona podría seguir paso a paso lo indicado en esta sección para replicar los resultados que se mostrarán luego. Por lo tanto, se debe ser lo suficientemente específico en explicar el diseño y procedimiento, sin dejar aspectos clave de lado.

