\section{Resultados}

La sección de resultados de una memoria o tesis presenta los hallazgos de la investigación sin interpretación, ofreciendo un relato claro y directo de los datos recopilados. Esta sección debe estar organizada lógicamente, siguiendo típicamente la secuencia de preguntas de investigación o hipótesis y la metodología empleada. 

Se comienza presentando los resultados en un orden lógico que se alinee con tus preguntas de investigación o hipótesis. Es recomendable utilizar tablas y figuras para mostrar los datos de manera clara y efectiva, asegurándose de que cada uno esté adecuadamente titulado, etiquetado y referenciado en el texto.

Para cada conjunto de datos presentado, es clave proporcionar una descripción concisa de los hallazgos. Es aconsejable incluir estadísticas relevantes, como medias, desviaciones estándar y valores $P$, para respaldar los resultados. Es importante asegurarse que la narrativa sea clara y refleje con precisión los datos presentados en las ayudas visuales.

La sección de resultados debe centrarse únicamente en informar lo que se encontró. Se debe dejar cualquier interpretación de estos hallazgos para la sección de discusión. El objetivo aquí es presentar los datos de manera tan objetiva como sea posible.

Aunque todos los hallazgos relevantes para tus preguntas de investigación deben incluirse, destacar los resultados más significativos es recomendable. Estos son los hallazgos que impactan directamente en la hipótesis, pregunta de investigación o problema a resolver.





