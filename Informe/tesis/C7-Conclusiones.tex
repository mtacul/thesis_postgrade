\section{Conclusiones}

Escribir las conclusiones de una memoria o tesis implica sintetizar los hallazgos, reflexionar sobre las implicaciones y sugerir direcciones para investigaciones futuras. Es crucial encapsular de manera breve la esencia de la investigación, demostrando cómo aborda los objetivos y contribuye al campo más amplio. 

Se comienza resumiendo de manera breve los hallazgos más significativos de la investigación. Se destaca cómo estos resultados responden a la pregunta de investigación o abordan la declaración del problema, vinculándolos directamente con los objetivos. Esta recapitulación debe ser clara y concisa, enfatizando las contribuciones del estudio.

Luego, se elabora respecto de las implicaciones de los hallazgos para el estado del arte o investigaciones futuras. Esto involucra interpretar la significancia de los resultados en un contexto más amplio, mostrando su relevancia y potencial impacto. Se puede resaltar cualquier nueva perspectiva, teoría o modelo que la investigación haya introducido o apoyado.

Además, se discute de manera transparente las limitaciones del estudio, incluyendo cualquier restricción que pueda influir en la generalización o aplicabilidad de los hallazgos. Esta honestidad aumenta la credibilidad de la investigación y ayuda a enmarcar el contexto en el que tus conclusiones deben ser interpretadas.

Se puede incluir propuestas para investigaciones futuras que se deriven de los hallazgos, preguntas sin respuesta o limitaciones. Esto no solo demuestra la naturaleza evolutiva de la investigación sino que también alienta a otros a construir sobre el trabajo realizado, indicando caminos potenciales para exploraciones adicionales.






