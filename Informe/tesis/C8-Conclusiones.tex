\section{Conclusiones}


Se ha completado un marco teórico detallado sobre el modelado del ambiente espacial y el sistema de determinación y control de actitud (ADCS) para CubeSats en misiones de observación terrestre, basado en estudios relevantes de la última década. Este marco teórico ha permitido validar los modelos empleados y asegurar su coherencia con investigaciones y especificaciones actuales de componentes comerciales, proporcionando una base sólida para implementar modelos de simulación realistas en el contexto de misiones CubeSat.

En cuanto al modelo del ambiente espacial, se ha utilizado el propagador SGP4, logrando representar con precisión las posiciones y velocidades orbitales del CubeSat bajo perturbaciones típicas de órbitas bajas. Adicionalmente, se implementaron modelos de sensores (magnetómetros y sensor solar) y actuadores (magnetorquers y ruedas de reacción), permitiendo simular las dinámicas de actitud del CubeSat y avanzando en la creación de una suite de simulación en condiciones realistas de operación.

La implementación del filtro de Kalman semiextendido ha sido exitosa, permitiendo estimaciones confiables de cuaterniones y velocidades angulares del CubeSat con un margen de error inferior al 5\% respecto al modelo dinámico. Esto constituye una herramienta eficaz para integrar las mediciones de múltiples sensores y actualizar las estimaciones de estado de manera continua, lo cual es fundamental para las tareas de apuntamiento y estabilidad del CubeSat.

En el control de actitud, se implementaron y probaron los controladores PD y LQR dentro de la suite de simulación. Tras una comparación bajo condiciones de simulación equivalentes, se seleccionó el controlador LQR por optimizar los parámetros de apuntamiento con mayor precisión y alcanzar la estabilidad deseada de manera más efectiva.

Con estos avances, se ha completado el hito de una suite de simulación lista para su funcionamiento sin optimización. Para analizar cómo se relacionan los componentes físicos con los parámetros de rendimiento de apuntamiento (MoP), se utilizaron datos del SUCHAI-3 junto con otros datos relevantes de geometría y parámetros orbitales. Se observó que el jitter y la precisión de apuntamiento están directamente relacionados con los sensores, mientras que la agilidad del sistema se asocia más estrechamente con el tiempo de asentamiento. Además, se confirmó que las ruedas de reacción ofrecen un mejor rendimiento que los magnetorquers, aunque a un costo más elevado.

Finalmente, se ha completado una revisión exhaustiva de las herramientas de optimización disponibles en scipy.optimize. El solver Powell demostró ser eficaz para encontrar un mínimo cercano al global, incluso en problemas no convexos y sin requerir información del Hessiano o Jacobiano, logrando resultados comparables a los de la grilla en dos tipos de problema. Esto hace que Powell sea una opción prometedora. Resta probar el entorno Pyomo, utilizando solvers más avanzados que puedan encontrar mínimos globales con mayor precisión.

Además, resta la verificación final de la suite de simulación, que corresponde al OE6, la cual será realizada mediante la comparación de los resultados de la simulación con datos empíricos del SUCHAI-3. Esto permitirá validar la precisión de la suite y evaluar la cercanía de sus resultados respecto a datos reales, asegurando así su aplicabilidad en misiones reales de observación terrestre. Además, el cumplimiento exitoso de los primeros cinco objetivos específicos establece una base sólida para el análisis de los datos empíricos y refuerza la capacidad de la suite para optimizar la selección de sensores y actuadores conforme a los SE envelopes definidos.