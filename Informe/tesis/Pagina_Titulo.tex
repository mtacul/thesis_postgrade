%\begin{titlepage}
	
% Primera fila con las imágenes en esquinas
\begin{minipage}{0.35\textwidth}
	\flushleft
	\includegraphics[height=50pt]{udec.png}
\end{minipage}
\begin{minipage}{0.65\textwidth}
	\flushright
	\includegraphics[height=90pt]{logo_FI.png}
\end{minipage}

\vspace{2cm} % Espacio vertical entre imágenes y contenido	
	
\begin{center}

{\Large
\textbf{Implementación de una suite de optimización para el apuntamiento de CubeSats de observación terrestre en órbitas bajas.\\}
}
\vspace{2.5cm}

{\large
\textbf{Matías Ignacio Tacul Vargas\\}
}
\vspace{2.5cm}

{\normalsize
Tesis presentada a la Facultad de Ingeniería de la Universidad de Concepción para optar al grado de Magister en Cs. de la Ingeniería mención Ingeniería Mecánica\\
}
\vspace{2.5cm}


{\normalsize
Profesores guía:\\
Dr.-Ing. Bernardo Hernández V. \\
Dr.-Ing. Alejandro López T.
}
\vspace{0.5cm}

{\normalsize
Octubre 2024\\
Concepción, Chile
}
\vspace*{\fill}




\end{center}

{\footnotesize
\copyright 2024 Matías Ignacio Tacul Vargas\\
Se autoriza la reproducción total o parcial, con fines académicos, por cualquier medio o procedimiento, incluyendo la cita bibliográfica del documento
}
\vspace{0.5cm}

%\end{titlepage}